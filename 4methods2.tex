\chapter{Methods}
\label{chapter:methods II}
 
\section{Random Reads in HBase}

Unlike some cloud-datastores that are optimized for random reads like PNUTS, HBase is write-optimized by using on-disk structure that can be maintained using sequential IO. Its records are never overwritten, instead, updates are written sequentially to new files in disk. That means that multiple updates of the same record will be spread over many files, so when reading it, multiple IO operations will be needed to merge the separate updates. On the other hand, as we already explained, all writing is sequential, so HBase excels at writting and consequently, in scans, which are sequential reads. This is a simple tradeoff between optimizing for reads and optimizing for writes.

\subsection{Random Reads in our heavy-write cluster}

In this section we have tested random reads for our HBase fully write-optimized cluster. In HBase, there is no big room for improving random reads, but still some little tuning can be done to achieve a better random read performance than the default one.
\par
Before starting, we must describe how our reads are going to be, whether they will request an entire row, that will be the darkest room for enhancing it, or they will ask for a little part, better scenario as HBase stores FamilyColumns in separated files and only a few will be required in order to return the result. 
\par
The use case we performance is fetching 1, 10, 25, 50 or more random video details at once. The row keys of these random videos are known before hand, so we only have to look for them and retrieve its details. In these reads, not all data is requested, but instead just the main data, which is within the first ColumnFamily (CF1).                  

\subsection{Studying random read performance}

Now that we have characterized reads, lets study a bit how we can enhance them. 
\\
First of all, the number of requested rows is really low if we compare with the total number of rows our data has, almost 45 millions counting duplicated ones. So we discard the idea of creating a \textit{Scan} instance (previously described in HBase background chapter) instead of \textit{Gets} objects, since it will be helpful if we were requesting a high number of rows or if the keys would cover a small key range, but the row keys we are looking for do not represent it and above all, they are not sequential ones, so they are not even in the same region. Therefore, using \textit{Scan} would not help. 
\par
About whether to use Gets or Scan methods, Lars George quantifies it specifying when it worths using one or the other. Its studies demostrate that translating many \textit{Gets} into a \textit{Scan+Filter} is beneficial if the \textit{Scan} would return at least 1\%-2\% of the total rows to the client \cite{http://permalink.gmane.org/gmane.comp.java.hadoop.hbase.user/33133}\. In our particular case, that we seek a maximum of 50 rows, it only represents 0.00064\% of the total rows without the duplicates, so using \textit{Gets} will be likely more efficient.
\par
Another keypoint to have in mind is that since the desired row keys are totally random, some times our lookups can be touching most of the regions and others just hitting the same region, in spite of the fact that our data is evenly distributed accross the regions. But this last thought is something we can not avoid, it depends upon the nature of the chosen row keys.
\par
One good idea could be using \textit{HBaseFilters} to limit the search. Filters allow to do fine-grained searches such as combing values bigger than X, rows with a certain timestamp, rows with a specified column, etc. But since all our rows have all columnFamiles completely filled and we merely want all the fields of a certain columnFamiliy there is no possibility to use filters to avoid some row seeks. Given the nature of our searches, we only need to look into a few HFiles, the ones containing the desired columnFamily. To narrow the scope of the search and thus speed up the process, we can use \textit{Get.addFamily()} method to just process the valid columnFamily files and no others. Its equivalent HBase filter would be \textit{FamilyFilter} which filters on the columnFamily, but it is better to use the prior method.
\par

An additional improvement we can make is batching \textit{Gets} objects, instead of sending one by one. Using it is as simple as running the \textit{multi-get} method HBase supplies. We just need to write all the \textit{Get} instances within a list and then call \textit{Htable.get(List<Get> gets)}. This method first sort the requests by RegionServer and then serially goes to the RegionServer to process the multi-get. It is done in a parallelized way across RegionServers. 
This method could be optimized by changing it to a multi-threading behaviour. Instead of going one by one RegionServer, it could sort the requests by RegionServer as before, and then could spin up as many threads as target RegionServers \cite{http://comments.gmane.org/gmane.comp.java.hadoop.hbase.user/34417}. But it does not worth for our little multi-get operation.
\bigskip
Using Hbase table as the source for a MapReduce job is discarded due to the little number of requested keys per examination. But of course it is another type of reading that HBase supports, which is incredibly useful in many situations. 
\par

\subsection{Proceding with random read}

Once we have studied what our searches are going to require and how we must act, we can proceed. Our first try is simple, we create a list of \textit{Gets} objects (List<Get> gets), each one with a random key row obtained from a prior insertion, and we add them the columnFamily we want to check by calling to \textit{Get.addFamily()} method for each one. Then we execute it, \textit{Htable.get(List<Get> gets)}, and we measure the time it takes to carry out.
\bigskip
Here we show the outcome:
\fixme{RESULTS}
\bigskip
Results are not that bad, but we can do some more improvements:
\begin{itemize}
\item Configuring block cache:
\par 
HBase has a built-in cache to improve read performance. It just left data blocks read from HFiles in a cache if there is enough room for it. It helps reducing disk IO.
\par
 This cache is configurable at columnFamily level, which means that user can choose which columnFamily can be settled in the block cache and which ones not, even user can choose between different cache priorities: \textit{In-Memory}, to try to keep the block in memory more aggresively, \textit{blockCache} = True, the block will be placed within the cache if there is room remaining, and \textit{blockCache} = False, blocks from that columnFamily will not be cached.
\par 
To leverage this feature, we have to change how our HBase table is created: 
\begin{itemize}
\item For the columnFamily CF1, which is the one we want to fetch data, we include \textit{blockCache} = true, and we set it to the highest priority with \textit{In-Memory} = true
\item The rest of the columnFamilies continues as before.
\end{itemize}

Without a doubt that if we just read once, and our requested data is not within the same block, it will be difficult to see how block cache helps, but in a scenario were we were continuosly retrieving 25 row keys per time, this feature will help for sure.


\item Bloom Filters:

HBase supports Bloom Filter \cite{BloomFilter http://en.wikipedia.org/wiki/Bloom_filter}. Bloom Filters are a mechanism to figure out whether an HFIle/storeFile stores a specific row, rowCol cell or not without loading the file and scanning the block. They avoid the step of going through each HFile's block index , which has the start row key of each block inside it, to check whether the row can be there or not, and if it does, then HBase needs to load the block and start scanning in order to confirm if the row is there or not. The drawback of Bloom Filter is that they have to be stored within the HFile and consequently, HFile's size will be boosted. 
\par
By default Bloom Filter is disabled. User just need to alter the table or create a new one adding the BLOOMFILTER => 'ROW' or 'ROWCOL' parameter to enable it. Bloom Filters are configurable at columnFamily level and within it, Bloom Filters can be at row or at row + column level. We set up it at row level because we are not looking for specific cells.
\bigskip

Our results:




\item Tuning HFIle's block size:
\par
HFile are the actual HBase storage files. Each one is composed of blocks which are the smallest unit of data HBase reads and places in the Block Cache. These blocks store key/value pairs and have a minimum size, which is by default 64KB. To achieve better performance, we can modified its minimum block size. If we want to improve random reads, we should decrease this value to avoid too many key/value pairs within each block, because the read operation always loads entire blocks and then look inside the block for the key/value, so setting it to a lower value will decrease the amount of data fetched by each seek operation, thus decreasing IO and time needed for decompression. On the other hand, it will require more memory to hold the block index, now bigger due to the rise in the number of blocks.
\bigskip
To get the best block size, we compute the average key/value size of our desired columnFamily (CF1) by using the \textbf{HFile tool} (\textit{org.apache.hadoop.hbase.io.hfile.HFile}). It allows us to go through the metadata of each HFile (see figure X.X) where the key/value average is displayed

\bigskip
\fixme{figure HFile Tool}
\bigskip

Thanks to it, now we can set the best block size for our columnFamily (CF1) and thus, we will achieve better random read performance.

 


\end{itemize}

SUBIR EL ARCHIVO PUT CON -DFG.BLOCKSIZE = 128...256 y listo calisto.


hbase.mapreduce.hfileoutputformat.blocksize
The mapreduce HFileOutputFormat writes storefiles/hfiles. This is the minimum hfile blocksize to emit. Usually in hbase, writing hfiles, the blocksize is gotten from the table schema (HColumnDescriptor) but in the mapreduce outputformat context, we don't have access to the schema so get blocksize from Configuration. The smaller you make the blocksize, the bigger your index and the less you fetch on a random-access. Set the blocksize down if you have small cells and want faster random-access of individual cells.








Meter lo de usar row keys menores etc...longint...en vez de string etc.
----------------------
Using bloom filter is almost mandatory there;
You might also want to try Short Circuit Reads and be sure you get 100\%
data locality (major compact your table first)
-------------------
describir mejor el dataset que tengo..que hay mil de videos repetidos como se puede ver al mirar reduce input groups compare to map output records.
-----
lo de los tres ttl (timestamp solo tres)

\fixme{la tabla de medeiros para test environment, violin-memory (http://www.violin-memory.com/wp-content/uploads/hadoop-benchmark.pdf?d=1)para sacar info de ycsb y read con filter de solo familia. Slurm que caracteristicas pido para mi cluster...}